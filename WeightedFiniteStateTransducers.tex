% for better visibility when space is not an issue
\begin{comment}
	\pagebreak
\end{comment}

\section{Weighted Finite State Transducers}
\begin{comment}
\subsection{Structure}

	\begin{itemize}
	
	\item WFSTs
	\begin{itemize}
	\item Big Picture
	\item Projectivity
	\item Decombosable Score Function
	\end{itemize}
	
	\item Floyd Warshal
	\begin{itemize}
	\item Floyd Warshal Recap
	\item Matrix Multiplication
	\item Semirings
	\item FW as MM
	\end{itemize}
	
	\item Finding Normalization Constant
	\begin{itemize}
	\item Computing infinite Sums
	\item Warshal-Floyd-Kleene
	\item Back to WFSTs
	\end{itemize}	
	\end{itemize}
\end{comment}

\begin{comment}
	\textbf{Weighted finite state transducer}: 
	Transducer maps sequences to another. They are used in Speech recognition, machine translation or transliteration (Map strings in one character set to another set).\\
	The transducers are finite state because we have a finite set of states, e.g. the non-terminals and terminals.
	They are weighted Weighted because we weight the probability of the transitions between the states.\\
\end{comment} 

\textbf{Finite state Transducer:} $T = \langle \mathcal{Q}, \Sigma, \Omega, \lambda, \rho, \delta \rangle$\\
\begin{comment}
	Finite set of states, input vocabulary, output vocabulary, function mapping state to initial scores, function mapping state to final scores, transition function mapping transitions to scores.\\
	There are usually a large number of accepting paths to form y from x, the mapping is therefore ambigous. We focus on the unambiguous case.\\
\end{comment} 

\textbf{Goal:} $p(y|x), x\in \Sigma^*, y\in \Omega^*$\\
\textbf{Score:} $score(\pi) = \lambda(q_{start}) + \delta(q_{start} | q_1) + ... + \rho(q_{end}) = \sum_n^{|\pi|} w(\tau_n)$. $p(y | x) = \frac{1}{Z} \sum_{\pi \in \Pi(x,y)} \exp (score(\pi))$\\
\begin{comment}
	$\pi$ denotes a sequence of state transitions for a sequence given input tokens x xand output tokens y.
	The normalizer Z is again infinite, since there may be loops in the finite state machine.\\
\end{comment}

$Z = \sum_{y' \in \Omega*} \sum_{\pi' \in \Pi(x,y')} \exp (\sum_n^{|\pi'|} score(\tau_n))$



\textbf{FW Algo:} $\forall i,j,k \text{ } if(dist[i][j] > dist[i][k] + dist[k][j])\\\text{ then } dist[i][j] \leftarrow dist[i][k] + dist[k][j]$\\
\begin{comment}
	Finds shortest paths between all pairs of vertices, thereby the transitive closure of a graph, in $O(n^3)$.\\
\end{comment} 

\textbf{MM-Semiring:} $\text{for } i \text{ to N}, \text{for } p \text{ to P} \\
\text{sum}\leftarrow \textbf{0}\\
\text{for } m \text{ to M}\\ 
\text{sum}\leftarrow \text{sum} \oplus (sum,A[n][m] \otimes B[m][p])\\
C[n][p]$\\
\begin{comment}
	In the second loop, we initialize a sum with $\bar{0}$, which is written to $C[i][j]$.
	Depending on the semiring, this sum captures different properties of the weighted graph.\\
\end{comment} 

\textbf{FW-MM:} $\forall i,j,k \text{ } \oplus (sum,W^{k-1}[i][k] \otimes W^{1}[k][j])$\\


\textbf{Kleene *:} $W^* = \oplus_{k=0}^\infty W^k = I+WW^* \Leftrightarrow W^* = (I-W)^{-1}$, semiring-closed if it has such a closure\\
Only $W^*$ we should know is $\frac{1}{1-X}$\\
\begin{comment}
	The last equation comes from the geometric series.\\
\end{comment} 

\textbf{Warshall-Floyd-Kleene:} $\forall i,j,k \text{ } dist[i][j] \leftarrow dist[i][j]\oplus (dist[i][k] \otimes dist[k][k]^* \otimes dist[k][j])$









